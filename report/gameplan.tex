\documentclass{article}
\usepackage{graphicx} % Required for inserting images
\usepackage[utf8]{inputenc}
\usepackage[margin=1.0in]{geometry}
\usepackage{amsmath}
\usepackage{amsfonts}
\usepackage{amssymb}
\usepackage{physics}
\usepackage{enumitem}                       % custom enum labels
\usepackage{parskip}          
\usepackage{geometry} 
\geometry{
  paperwidth=18cm,
  left=8mm,
  right=8mm,
  top=8mm,
  bottom=8mm,
}
\title{Topological Qubit Simulations in t-junction Kitaev Networks}
\author{Hrishikesh Belagali}
\date{\today}
\newcommand{\ds}{\displaystyle}
\newcommand{\pfn}[1]{\textrm{#1}}  % enables new functions
\newcommand{\mbf}[1]{\mathbf{#1}}  % mathbf
\newcommand{\C}{\mathbb{C}}        % fancy C
\newcommand{\R}{\mathbb{R}}        % fancy R
\newcommand{\Q}{\mathbb{Q}}        % fancy Q
\newcommand{\Z}{\mathbb{Z}}        % fancy Z
\newcommand{\N}{\mathbb{N}}   
\newcommand{\K}{\mathbb{K}}  % fancy N
\newcommand{\V}{\mathbf{V}} %vector space 
\newcommand{\0}{\mathbf{0}} %zero vector 
\newcommand{\from}{\leftarrow}
\renewcommand{\i}[1]{\textit{#1}}
\renewcommand{\b}[1]{\textbf{#1}}
\newcommand{\qed}{$\square$}
\begin{document}

\maketitle

\section{Introduction}
In this work, I aim to simulate a topological qubit through the quantum braiding of 3 Kitaev chains, using Python and NumPy.
The goal of doing so is to gain some knowledge on the nuances of topological quantum computing, and a better understanding of non-Abelian statistics.
\section{Formalism}
\subsection{The Physical Model}
\subsubsection{Kitaev Chains}
A Kitaev chain is a 1-dimensional topological superconductor, which can host Majorana Zero Modes (hereafter abbreviated as MZMs). In terms of normal fermionic operators, the Kitaev chain Hamiltonian is given by
$$\hat H = -\mu \sum_{j=1}^N c_j^{\dagger}c_j - t\sum_{j=1}^{N-1} \pqty{c_j^{\dagger}c_{j+1} + c_{j+1}^\dagger c_j} + \Delta \sum_{j=1}^{N-1} \pqty{c_jc_{j+1} + c^{\dagger}_{j+1} c_j^{\dagger}}$$ 
where - 
\begin{itemize}
    \item $\mu$: chemical potential 
    \item $t$: hopping amplitude
    \item $\Delta$: p-wave superconducting pairing
    \item $c_j, c_j^{\dagger}$ fermionic creation/annihilation operators
\end{itemize}
Majorana modes appear at the ends of the chain in the topological phase when $\abs{\mu} < 2t$.
We can express the Hamiltonian in the more convenient Majorana operator form. These operators are defined as
$$\gamma_{A, j} = c_j + c_j^{\dagger}$$
$$\gamma_{B, j} = -i(c_j - c_j^{\dagger})$$ 
The convenient Hamiltonian becomes
$$\hat H = -\frac{\mu}{2}\sum_{j=1}^N \pqty{1 + i \gamma_{A, j}\gamma_{B, j}} + \frac{i}{2}\sum_{j=1}^{N-1} (t - \Delta) \gamma_{B, j}\gamma_{A, j+1} + \frac{i}{2}\sum_{j=1}^{N-1}(t + \Delta)\gamma_{A, j}\gamma_{B, j+1}$$
\textbf{Milestone: Successfully simulate a Kitaev chain and identify MZMs (ideally through visual and mathematical representation)}
\subsubsection{t-Junction Kitaev Network}
The t-junction Kitaev network is the minimal geometry required to host MZMs. They will be simulated as tight-binding lattices, and the full network will be modeled as a graph of Kitaev chains, where coupling terms connect the adjacent sites and 
time-dependent parameters are used to control MZM motion. This will be essential for the braiding process.

\textbf{Milestone: Identify time-dependent MZM movement and visually represent the Kitaev network as a graph.}
\subsection{Simulation Technique - BdG Formalism}
\subsubsection{The Wavefunction}
The Bogoliubov-de Gennes (BdG) formalism will be used to simulate the full quantum state evolution. 
$$i\dv{t} \Psi(t) = \hat H(t) \Psi(t)$$ 
where $\Psi(t)$ is the BdG quasiparticle wavefunction. 
The braiding (which is inherently adiabatic) will be simulated by evolving the Hamiltonian parameters slowly in time. 
\subsubsection{Adiabatic Quantum Braiding}
\begin{itemize}
    \item Create a terminal MZM in each arm. 
    \item Tune parameters to move one MZM through the junction into the other arm.
    \item Continue moving until the MZM has encircled another MZM, this is a braid. 
    \item Ensure the evolution is adiabatic to stay in the ground-state.
\end{itemize}
Track the braiding by projecting the final state back onto the degenerate ground-state basis and compute the overlap to find the unitary braid transformation. Project the Hamiltonian into the Majorana basis, and find the zero-energy eigenstates. 
Plot their spatial localization to see where the MZMs are during the simulation.
\subsection{Topological Protection}
Simulate the following errors - 
\begin{itemize}
    \item Local Perturbations 
    \item Pauli Errors
\end{itemize}
Measure fidelity of the error-states, entanglement entropy, parity conservation and other metrics. Ideally, you will show that only Majorana pair fusion can actually destroy information.
\end{document}